\documentclass[12pt,letterpaper,oneside,twocolumn]{article}
\usepackage{amsmath,amssymb}
\usepackage{fullpage}
\title{Inflated Fabric Habitat for Martian Surface and Nanoscale Life Support Catalysis} % or use an ampersand?
\date{\today}
\author{Brent Carter, Grant Gilliam, Zack Hester, Mark Kaufman, Kasey Orrell, Alex Ray, \and Kris Tesh\\ Advisors: Russel Gorga, PhD; Warren Jasper, PhD; and Andre Mazzoleni, PhD\\ North Carolina State University, Colleges of Engineering and Textiles}

\begin{document}
\maketitle

\begin{abstract}
This project focuses on two key components of a manned mission to Mars: habitat systems and life support systems. These two areas of mission design have presented major obstacles in the past. Our research involves analyzing these areas from a different perspective and providing innovation through the use of state-of-the-art materials and technologies.

Space missions involving a prolonged stay on an extraterrestrial surface necessitate the use of a habitation system. We will present a design for a multi-layer fabric-based inflatable structure. This system will include a flexible airlock, modular radiation shield pods, and an advanced air control and filtration system. 

The second component of this project focuses on life support systems.  Current life support systems that utilise the Sabatier reaction use macroscale metallic pellets as their catalyst to combine hydrogen and carbon dioxide in order to produce methane and water. Nanoscale catalysis is an emerging field, with much of the studies focusing on how the efficiency of catalyzed reactions can be improved by higher surface area catalyst materials. Our research investigates the effects that catalyst surface area has on the Sabatier reaction. When using nanoscale particles versus macroscale particles, you will inevitably decrease the total mass of catalyst material. As a result we will compare the effects of an increase in surface area with the effects of a decrease in total catalyst mass in order to optimise efficiency.
\end{abstract}

\section{Introduction}
Manned exploration of our solar system has been stalled since mankind first walked on the moon forty years ago. While many probes, robots, and satellites have been launched to gather scientific data, people have not ventured far beyond our planet. The obvious next step has always been to send humans to Mars, but the difficulties of a manned mission to the Red Planet are daunting. Development of the necessary technologies for keeping a crew of astronauts alive for the lengthy trip and surface stay is an as-yet unanswered challenge, making human safety and life support an important focus of any mission planning.

\section{Requirements}
There are many obstacles to overcome when planning for human exploration of Mars, most of which are a result of the hostile environment offered by deep space and the planet’s surface. Any successful mission will have to provide a safe living space for the four-person crew during their minimum thirty-day stay on Mars. This involves dealing with several inhospitable conditions such as radiation exposure, low atmospheric pressure, temperature variation, lack of available water and oxygen, and several secondary concerns such as micrometeorite impacts and dust storms. The mission allows for five launches, each with a payload weight of 125 metric tons. All equipment must fit inside a ten meter payload shroud.

Any viable surface habitat must provide the following:
\begin{itemize}
  \item Shielding from radiation, including Solar Proton Events (SPE) and Galactic Cosmic Rays (GCR)
  \item Stable pressure and temperature within the habitable range
  \item A breathable atmosphere and access to food and water
\end{itemize}


Important parameters to consider include
\begin{itemize}
  \item Keeping the launch weight as low as possible
  \item Making the construction and/or assembly as simple as possible
  \item Ensuring that all materials can be stored in the limited volume of a spacecraft
  \item Providing both longevity and reliability in life-support systems.
\end{itemize}

\section{Design}
\subsection{Surface Habitat}
Many proposed expeditions to the Martian surface suggest using the lander as the habitat, mainly because of the low volume-to-weight ratio of re-use and the relative simplicity of reusing equipment. However, this approach does not allow for a permanent presence on Mars and also makes the problem of designing the lander much more difficult, as it has to incorporate surface habitat facilities. Another practical option for a rigid structure is to build from prefabricated units. The main issue here is the difficulty of construction, which would require equipment to move the structural components.  This equipment is often complex and heavy.

For these reasons, inflatable structures show promise. Due to the low atmospheric pressure on Mars, an inflatable habitat can accommodate a large enough pressure differential to maintain structural support while providing significant advantages in launch weight and volume reduction. Construction involves pressurizing the interior and anchoring the structure to the surface, avoiding any complicated machinery or assembly that would be difficult for astronauts to accomplish in spacesuits.

Having a curved dome shape is desirable when dealing with micrometeorite impacts. Without an atmosphere to burn up objects on reentry, Mars is hit by many tiny meteorites that would never reach the surface on Earth. These impacts cannot be avoided with any degree of certainty, as they occur across the entirety of the planet at fairly regular intervals. However, the damage can be minimized by presenting a curved surface, decreasing the probability of a head-on strike (ref. 4). The curved outer surface also prevents aerodynamic issues with the fast (though low-density) Martian winds (ref. 2).

Crew living space, however, benefits most from a disc-shaped design. This shape provides more usable area than a dome as it does not require tiered floors or walkways to make the best use of cabin volume. Rather than examine the trade-offs between the dome shape and the disc shape, a better design simply nests the disc inside the dome. This has several added advantages, including reducing the amount of pressure pushing on the outer dome layer (lessening its tensile strength requirements). This also provides impact redundancy; should a micrometeorite penetrate the outer dome (mainly needed for radiation shielding), there is an entirely separate layer keeping the pressure and breathable atmosphere inside the living quarters. This increases crew safety while patching the hole. The entire inflatable habitat is also cheaper and easier to build with this design, mainly because the materials can be customized for their specific function.

This will result in designing three major material layers: an outermost modular radiation protection layer, a low-pressure (20 kPa) dome layer, and an innermost high-pressure (50 kPa) cabin layer. Each layer consists of a combination of materials with different useful properties, including Vectran for tensile strength, gilded BoPET for UV protection, and Demron for radiation attenuation. The outermost layer (which is primarily for radiation protection) will also make use of Martian regolith via modular shield pods to be filled for long-duration Mars missions. In order to test the capabilities of each layer, example swatches will be made and put through two types of experiments. First, tensile strength tests will be done to insure the materials can withstand the forces caused by the high pressure differential. And second, each material’s radiation attenuation will be characterised through the use of a gamma ray emitter.

Finally, a flexible structure allows for the design and use of an inflatable airlock.  Current airlock technology involves pumping air out of a sealed rigid structure.  On the Space Shuttle, 10\% of the air in the airlock is lost each opening cycle, representing a drain on oxygen and buffer gases in the life support system (ref. 3). We will develop a prototype two-person inflatable airlock that will take advantage of the fabric technology used in the habitat. For any long-term mission, this improved airlock will greatly reduce losses of non-replenishable gases while maintaining the safety provided by a two-person design.

\subsection{Sabatier Reactor Catalyst}
Launching a mission to Mars with bulk supplies of oxygen and water is infeasible due to weight and space constraints. Current studies on the longevity and reliability of reclamation and filtration techniques show that these processes alone are not sufficient to provide these necessary resources.  Furthermore, the atmosphere on Mars is thin and has very little oxygen. Carbon dioxide, however, is abundant comprising 95\% of the atmosphere (ref. 1). It then becomes an obvious solution to convert this carbon dioxide into water and oxygen. Two of the most commonly discussed chemical reactions that utilize carbon dioxide as a reactant to produce water are the Bosch reaction and the Sabatier reaction. The Bosch reaction is problematic because of carbon buildup, but the Sabatier reaction has been tested and examined by NASA for decades.

The basic chemical equation of the Sabatier reaction is as follows:
\[ CO_2 + 4H_2 \xrightarrow{Catalysis} CH_4 + 2H_2O + heat.\] % actual energy calc would be nice
The bulk carbon dioxide from the Martian atmosphere (or from respiration on-board the transit vessel) is mixed with hydrogen brought from Earth and run through a tube filled with a catalyst (usually nickel or ruthenium). The reaction is exothermic, and produces methane and water. The methane can then be used for fuel, either as a return launch propellant or for heating the habitat. The water can either be used for human consumption or electrolyzed to extract diatomic oxygen (for breathing) and hydrogen (to go back into the reactor). We believe this reaction is the best choice for water and oxygen production on the Martian surface.

A basic Sabatier reactor is a simple tube or pipe filled with spherical pellets. Since catalyst efficiency is related to surface area, alternate catalyst designs that maximize or increase surface area might improve overall reaction performance, with the ultimate goal of either miniaturizing the reactor for portable applications (as on a spacesuit) or facilitating bulk use to replenish a habitat. Size and weight are always extremely important properties to control in space systems, and a reactor’s size and weight could potentially be minimized through optimized nanoscale catalysis, without reducing output efficiency. 

Our research explores this idea of nanoscale catalysis with the use of nickel nanoparticles. These nanoparticles will be deposited onto a microscale substrate. The substrate we have chosen to use is a fused silica quartz microfiber. These microfibers are on average about 4 microns in diameter and prove to be a very resilient substrate while still providing a large surface area for the deposition of the nanoparticles . This type of catalyst will be tested in a small-scale Sabatier reactor and compared to a nickel pellet catalyst control to determine what effect, if any, the new nanoscale design has on the reaction.


\section{Conclusion}
The greatest obstacle to manned missions in the space program is ensuring the safety of the astronauts involved. Improvements in habitat design and life support systems would be important steps towards demonstrating the feasibility of a human mission to Mars. The quantitative and qualitative analysis of a simple and safe inflatable fabric-based habitat system, as well as an analysis of the advantages that nanoscale catalysis provides in life support reactors are important components in such improvements and vital in any comprehensive study on manned Mars mission technology. 

\section{References}

1. Costard, François; Forget, François; and Lognooné, Philippe. Planet Mars: Story of Another World. Praxis Publishing, Ltd. 2008. pp 121-169. (Translated from original French by Bob Mizon.)
2. Fisher, Gary C. (And members of the Independence Chapter of the Mars Society). “Torus or Dome: Which Makes the Better Martian Home?” Bryn Athyn, Pennsylvania, 1999.
3. Hanford, Anthony J (Ph.D.). “Advanced Life Support Baseline Values and Assumptions Document.” NASA, Lockheed Martin Space Operations, Houston, Texas. August 2004. pp 21, 30, 33, 35, 39, 40, 43, 138.

4. Crossman, Frank (Ph.D.); Zubrin, Robert (Ph.D.).  On to Mars volume 2, Exploring and Settling a New World. Apogee, 2005.  pp 95. 





\end{document}






